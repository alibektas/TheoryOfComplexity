\documentclass{article}


\author{Ali Bektas}
\title{Complexity Theory - Notes}

\usepackage{amsmath}
\usepackage{amssymb}
\usepackage{amsthm}



\usepackage[toc , page ]{appendix}
\usepackage{hyperref}
	\hypersetup{
	    colorlinks=false,
	    linkcolor=blue,
	    filecolor=green,      
	    urlcolor=red,
	}
	\urlstyle{same}

\newtheorem{theorem}{Theorem}
\newtheorem{definition}{Definition}[section]
\newtheorem{corollary}{Corollary}[theorem]
\newtheorem{lemma}[theorem]{Lemma}

\def\mathify#1{\ifmmode{#1}\else\mbox{$#1$}\fi} % guarantee math mode
\newcommand{\ceil}[1]{\mathify{\left\lceil {#1}\right\rceil}}
\newcommand{\floor}[1]{\mathify{\left\lfloor {#1}\right\rfloor}}


\begin{document}
	\maketitle
	\newpage
	\tableofcontents
	\newpage


	\section{Grundlegendes}
	
	\begin{definition}
		Für eine Sprache $A \subset \Sigma^* $ ist die charakteristische Funktion $\chi_A \colon \Sigma^* \rightarrow \{0, 1\}$ wie folgt definiert:
		\[
			\chi_A = \begin{cases}
					1 &, x \in A \\
					0 &, x \notin A.
				\end{cases}
		\]
	\end{definition}

\section{Grundlegende Beziehungen}
	\begin{definition} \label{complexity_function}
		Eine monotone Funktion $f \colon \mathbb{N} \rightarrow \mathbb{N}$ heißt \textbf{echte Komplexitätsfunktion} falls es einen Transducer M gibt mit\\
		\begin{itemize}
			\item $M(x) = 1^{f(|x|)}$
			\item $space_M(x) = O(f(|x|))$ und
			\item $time_M(x) = O(f(|x|)+|x|)$.
		\end{itemize}
	\end{definition}
	

\section{Hierarchiesätze}
	
	
	\subsection{Unentscheidbarkeit mittels Diagonalisierung}

	Wir verwenden eine Kodierung , um die TMs zu kodieren. Die 
	Kodierung einer TM M wird durch $ \langle M \rangle $ bezeichnet. Wir können auch jedem Binärstring $w$ eine TM $M_w$ wie folgt zuordnen:
		\[ M_w = \begin{cases}
				M &, \langle M \rangle = w \\
				M'&, sonst.
				\end{cases}
		  \]
  	M' ist dabei eien beliebig aber fest gewählte TM. Für $M_w$ schreiben wir auch $M_i$ , wobei $i$ die Zahl mit der Binärdarstellung 1$w$ ist.

  	\begin{theorem}[Die Unentscheidbarkeit der Diagonalsprache] \label{UderDS}: Diagonalsprache ist semi-entscheidbar aber nicht entscheidbar.
  		\[ 
  			D = \{ \langle M_i \rangle | M_i  \textrm{ist eine DTM , die die Eingabe} x_i \textrm{akzeptiert.} \} 
		\]

  		Hierbei ist 
  		\[ 
  			x_1 = \epsilon x_2=0 , x_3=1 , x_4=00 , \dots 
		\] 
		die Folge aller Binärstring in lexikografischer Reihenfolge.\\

		Angenommen , wir haben eine Tabelle wo alle semi-entscheidbaren Sprachen drin sind und die Spalten dieser Matrix aus aller Wörter in lexi. Reihenfolge bestehen. Wenn $\bar{D}$ semi-entscheidbar wäre , würden wir sie in dieser Matrix finden also : 
		\[
			L(M_d) = \bar{D} 
		\]

		Da diese Sprache Komplement zu der Diagonalen dieser Matrix ist ist muss es einen Eintrag geben wo es widersprüchlich zu sein scheint. Das ist die Idee.

	\end{theorem}

	\begin{theorem}
		:Für jede berechenbare Funktion $g \colon \mathbb{N} \leftarrow \mathbb{N} $ existiert eine Sprache $D_g \notin DTIME(g(n))$.

		Sei\\
		\[
			D_g = \{ \langle M_i \rangle | M_i \textrm{ist eine DTM , die die Eingabe} x_i \\ \textrm{in} \leq g(|x_i|) \textrm{Schritten akzeptiert.} \}
		\]

		$D_g$ ist entscheidbar : Prüfe ob $x_i$ mittels $M_i$ in $ \leq g(|x_i|)$ Zeit entscheidbar ist.\\

		\dots Unter der Annahme , dass $D_g \in DTIME(g(n))$ ist , existiert eine $g(n)$-zeitbeschränkte DTM $M_d$ die das Komplement von $D_g$ entscheidet.

		\[
			L(M_d) = \bar{D_g} 
		\]

		Die Idee ist also dieselbe wie im obigen Satz.
	\end{theorem}

	\begin{theorem}[Gap-Theorem]
		Es gibt eine berechenbare Funktion $g:\mathbb{N} \rightarrow \mathbb{N}$ mit

		\[
			DTIME(2^{g(n)}) = DTIME(g(n))
		\]

		
		Wir definieren $g(n) \geq n+2$ so , dass für $2^{g(n)}$-zeit. DTM M gilt: \footnote{Warum ist dies wichtig?.}
		\[ 
			time_M(x) \leq g(|x|) \textrm{für fast alle Eingabe} x. 
		\]

		Wir definieren ein Prädikat

		\[
			P(n,t) : t \geq n+2 \text{ und für } k= 1, \dots , n \text{ und alle } x \in \Sigma_k^n \text{ gilt: }
			time_{M_k}(x) \notin \lbrack t+1 , 2^t \rbrack.
		\]
		
		\marginpar{Ob alle Wörter der Länge n durch k verschiedene Maschinen  \textbf{nicht} in $[t+1,2^t]$ Zeit erkannt werden.}
		
		Hierbei bezeichnet $\Sigma_k $ das Eingabealphabet von $M_k$. Da für jedes n alle. Da für jedes n alle

		\[
			t \geq max \{ time_{M_k}(x) | 1 \leq k \leq n , x \in \Sigma_k^n , M_k(x)  \textrm{hält} \}
		\]

		das Prädikat P(n,t) erfüllen , können wir g(n) wie folgt induktiv definieren :

			\[ 
				g(n) = 
					\begin{cases}
						2 &, n=0 \\
						min\{ t \geq g(n-1) + n | P(n,t) \} &, n > 0 
					\end{cases} 
			\]

		Da P entscheidbar ist , ist g berechenbar.

		Um zu zeigen , dass jede Sprache $ L \in DTIME(2^{g(n)})$ bereits in DTIME(g(n)) enthalten ist , sei
		$M_k$ eine beliebige $2^{g(n)}$-zeitbeschränkte DTM mit $L(M_k) = L$. Dann muss $M_k$ alle Eingaben x der Länge $n \geq k $ in Zeit $time_{M_k}(x)\leq g(n)$ entscheiden da andernfalls , P(n,g(n)) wegen $time_{M_k}(x) \in [g(n) + 1 , 2^{g(n)}] $ verletzt wäre. Folglich ist $L \in DTIME(g(n))$ da die endlich vielen Eingaben x der Länge n < durch table-lookup in Zeit $n+2 \leq g(n)$ entscheidbar sind.

	\end{theorem}
	
	\subsection{Zeit- und Platzhierarchiesätze}
	
	Um $D_g$ zu entscheiden , müssen wir einerseits die Zeitschranke $g(|x_i|) $ berechnen und andererseits $M_i(x_i)$ simulieren. Wenn wir voraussetzen dass g eine echte Komplexitätsfunktion (s. \ref{complexity_function} ) ist , lässt sich $g(|x|)$ effizient berechnen. Für die zweite Aufgabe benötigen wir eine möglichst effiziente universelle TM.
	
	\begin{theorem}[Die Simulierung von $M_i(x_i)$ bei einer univ.  TM] \label{35}
	Es gibt eine universelle 3-DTM U , die für jede DTM M und jedes $x\in {0 ,1}^*$ bei Eingabe $\langle M , x \rangle$ eine Simulation von M bei Eingabe in Zeit $O(|\langle M \rangle|(time_M(x))^2) $ und Platz $O( | \langle M \rangle | space_M(x))$ durchführt und dasselbe Ergebnis liefert.
	\end{theorem}
	\begin{proof}
		Betrachte folgende Offline-3-DTM U : \\
	\par{Initialisierung:}U überprüft bei einer Eingabe $w\#x$ zuerst , ob $w$ die Kodierung $\langle M \rangle$ k-DTM M ist. Falls ja , erzeugt U die Startkonfiguration $K_x$ von M bei Eingabe x, wobei sie die Inhalte von k übereinander liegenden Feldern der Bänder von M auf ihrem 2. Band in je einem Block von kb , \\$b=\ceil{log_2(||Q|| + ||\Gamma|| + 6)} $ , Feldern speichert und den aktuellen Zustand von M zusammen mit den gerade von M gelesenen Zeichen auf ihrem 3. Band notiert. Hierfür benötigt M' Zeit $O(kbn) = O(n^2)$
	\par{Simulation:} U simuliert jeden Rechenschritt von M wie folgt: Zunächst inspiziert U die auf dem 1. Band gespeicherte Kodierung von M , um die durch den Inhalt des 3. Bands bestimmte Aktion von M zu ermitteln. Diese führt sie sodann auf dem 2. Band aus und aktualisiert dabei auf dem 3. Band den Zustand und die gelesenen Zeichen von M. Insgesamt benötigt U für die Simulation eines Rechenschrittes von M  Zeit $O(kbg(n)) = O(n \cdot g(n))$ 
	\end{proof}
	
	\begin{corollary}{Zeithierarchiesatz\\}
	Für jede echte Komplexitätsfunktion $g(n) \geq n+2$ gilt: 
	\[
	DTIME(n \cdot g(n)^2) - DTIME(g(n)) \neq \emptyset
	\]
	\end{corollary}

	\begin{proof}
		G.z.z : $D_g$ ist für jede Komplexitätsfunktion \ref{complexity_function} $g(n) \geq n+2 $ in Zeit $O(n \cdot g^2(n))$ entscheidbar. Betrachte folgende 4-DTM M'. M' überprüft bei einer Eingabe x der Länge n zuerst , ob x die Kodierung $\langle M \rangle $ einer k-DTM M ist. Falls ja , erzeugt M' auf dem 4. Band den String $1^{g(n)}$  in Zeit $O(g(n))$ und simuliert M(x) wie im Beweis von Theorem \ref{35}. Dabei verminder M' die Anzahl der Einsen auf dem 4. Band nach jedem simulierten Schritt von M(x) um 1. M' bricht die Simulation ab , sobald M stoppt oder der Zähler auf Band 4 den Wert 0 erreicht. M' hält genau dann im Zustand $q_{ja}$ wenn die Simulation von M im Zustan $q_{ja}$ endet. Nun ist leicht zu sehen , dass M' $O(n\cdot g(n)^2)$-zeitbeschränkt ist und die Sprache $D_g $entscheidet. 	
	\end{proof}

	\begin{corollary}
		\[
		 	P \subsetneq E \subsetneq EXPSPACE
		\]
	\end{corollary}

	\begin{proof}
		Folgt unmittelbar aus dem vorigen Korollar
	\end{proof}

	\begin{theorem}
		Sei $f(n) \geq n+2$ eine echte Komplexitätsfunktion und gelte

		\[
			\limsup\limits_{n\rightarrow\infty} \frac{g(n) \cdot log g(n)}{f(n)} = 0 
		\]

		Dann ist

		\[
			DTIME(f(n)) / DTIME(g(n)) \neq \emptyset
		\]

		Für $g(n)=n^2$ erhalten wir beispielsweise die echten Inklusionen $DTIME(g(n)) \subsetneq DTIME(f(n))$ für die Funktionen $f(n)=n^2 , n^2log^2n$.
	\end{theorem}

	\begin{theorem}{Platzhierarchiesatz}: Sind $g(n),f(n) \geq 2$ und ist $f$ eine echte Komplexitätsfunktion \label{complexity_function} mit

		\[ 
			\liminf\limits_{n \rightarrow \infty} \frac{g(n)}{f(n)} = 0
		\]

		dann ist 

		\[
			DSPACE(f(n))/DSPCAE(g(n)) \neq \emptyset
		\]

		Damit lässt sich im Fall $g(n) \leq f(n) $ die Frage , ob die Inklusion von $DSPACE(g(n))$ in $DSPACE(f(n))$ echt ist , eindeutig beantworten: Sie ist genau dann echt , wenn $\liminf\limits_{n \rightarrow \infty} g(n)/f(n) = 0  $ ist , da andernfalls $f(n) = O(g(n)) $ ist und somit beide Klassen gleich sind. 
	\end{theorem}

	\begin{corollary}
		\[
		L \subsetneq L^2 \subsetneq DCSL \subsetneq CSL \subsetneq PSPACE \subsetneq ESPACE \subsetneq EXPSPACE.
		\]
	\end{corollary}

	\section{Reduktionen}

	\subsection{Logspace-Reduktionen}

	\begin{definition}[Logspacereduktion] Seien A und B Sprachen. A ist auf B \textbf{logspacereduizerbar} falls eine Funktion $f \in FL$ existiert , so dass für alle $x \in \Sigma^*$ gilt , 
		\[
			x \in A \iff f(x) \in B.
		\]
	\end{definition}

	\begin{lemma}
		\[ FL \subset FP \]
	\end{lemma}
	\begin{proof}
		Sei $f \in FL $ und sei M ein logaritmisch platzbeschränkter Transducer der f berechnet. Da M bei einer Eingabe der Länge n nur $2^{O(log(n))}$ verschiedene Konfigurationen einnehmen kann , ist M dann auch polynomiell zeitbeschränkt.
		
	\end{proof}
		
\end{document}