\documentclass{article}

\usepackage{amsmath}
\usepackage{listings}
\title{Übungsblatt 3}
\author{Ali Bektas 588063 \and Julian Kremer 562717 \and Ruben Dorfner 550204}


\begin{document}

	

	\maketitle

	
	\section*{Aufgabe 21}
		\subsection*{a)}
			z.z : K ist \textbf{E}-vollständig.

			\subsubsection*{K ist E-hart}
				Sei L eine bel. Sprache mit $L \in \textbf{E}$. Da L eine Sprache in \textbf{E} ist gibt es eine DTM M. Dann gilt: 

					\[
						\exists c \forall x \in L time_M(x) \leq 2^{c \cdot |x| + c}
					\]

				Sei M' eine andere DTM die f berechnen soll. M' bekommt die Eingabe von M als ihre Eingabe und dann schreibt sie $\langle M \rangle \# x \# 1^{c \cdot |x| + c}$ auf das Ausgabeband. Es ist offenbar dass $f \in \textbf{FL}$ denn das Erstellen von der Sequenz von Einsen keine Berechungsschritte am Arbeitsband benötigt. 

			\subsubsection*{$K \in E$ })
				Wenn K in E liegen sollte , dann müsste es eine DTM geben , die E-zeitbeschränkt ist. 
				Betrachte hierzu eine universelle 5-DTM U.Die ersten 3 Bänder haben dieselben Arbeitweisen wie sie im Satz 35 beschrieben sind. Das vierte Band wird zunächst mit den Einsen  , die die Zeitschranke von M bei Eingabe x beschreiben , befüllt. Das fünft Band dient als Binärzähler. 
				Bei jeder Aktualisierung (damit gemeint das , was im Satz 35 beschrieben ist.) bewegt sich der Lesekopf am 4. Band.Wenn Binärzähler eine neue Stelle am Band erstellt , bewegt sich der Lesekopf am 5-ten Band ein Schritt nach links.

				Nach Satz 35 gilt dann:
				
				\begin{align*}
					time_U(x) &\in 
					O(| \langle M \rangle | (time_M(x))^2) &\subset
					O(| \langle M \rangle | (2^(c \cdot |x| + c))^2) &\subset
					\textbf{E}
				\end{align*} 

		\subsection*{b)}
			Sei S eine K-harte Sprache.

			\subsubsection*{$\rightarrow$}
				Sei L eine beliebige Sprache mit $L \in E$ und $L \leq_m^{log} K$.
				Betrachte folgende Aussage:
					\[
					K-hart \iff \forall L \in E \exists f \in FL \forall x \in L x \in L \iff f(x) \in K.
					\]

				Wir nehmen eine beliebige Sprache $L \in EXP\textbackslash E$. Wir benutzen die Konstruktion aus Aufgabe 19 und dadurch wandeln wir eine beliebige Sprache L zu einer entsprechenden Sprache $L' \in \textbf{E}$
				um. Die Umwandlungsfunktion ist offenbar in \textbf{FL}.

				Da es eine E-harte Sprache gibt , sind wir fertig.
			\subsubsection*{$\leftarrow$}
				Trivial.
		\subsection*{c)}
			Wir nehmen eine der Zielsprachen aus (b) und entfernen die \#-Zeichen. Die Sprache ist dann wieder in \textbf{EXP.}

	\section*{Schluss}
		Wir sind fertig denn E ist nicht unter $\leq_m^{log}$ abgeschlossen wobei NP ist.




\end{document}