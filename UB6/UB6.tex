\documentclass{article}

\usepackage{amsmath}
\usepackage{amssymb}
\usepackage{german}

\def\mathify#1{\ifmmode{#1}\else\mbox{$#1$}\fi} % guarantee math mode
\newcommand{\ceil}[1]{\mathify{\left\lceil {#1}\right\rceil}}


\title{Übungsblatt 6}
\author{Ali Bektas 588063 \and Julian Kremer 562717 \and Ruben Dorfner 550204}


\begin{document}

	\maketitle
	
	\section*{Aufgabe 31}
		\subsection*{a)}
			Wir reduzieren VC auf DOMSET.
			\begin{enumerate}
				\item Sei $G=(V,E)$ein \textbf{ungerichteter} Graph und sei $G'=(V,E')$ ein \textbf{gerichteter} Graph.
				\item Für jedes $a,b \in V$ mit $(a,b)\in E$ enthalte $E' \cap = \{(a,b),(b,a)\}$
				\item Für jeden Knoten $a$ für den es gilt : $\neg\exists (a,b)\in E$ enthalte $E' \cap = \{(b,a)\}$ für alle $b \in V/\{a\}$
			\end{enumerate}

		\subsection*{b)}
			In einem Turniergraph mit n Knoten gibt es $ \frac{n\cdot(n-1)}{2}$ Kanten. Im Durchschnitt gibt es also $ \frac{(n-1)}{2}$ ausgehende Kanten für jede Knoten. Wir sammeln die DOMSET von einem Graphen wie folgt. 

			\begin{enumerate}
				\item Wir nehmen den Knoten , der mindestens $\frac{n-1}{2}$ ausgehende Kanten hat.
				\item Wir nehmen ihn und alle von ihm dominierten Knoten heraus.
				\item Wir wiederholen Schritt 1 für den übriggebliebenen Graphen. 
			\end{enumerate}

			Wir nehmen jedes Mal $\underbrace{1}_{\text{gewählter Knoten}} +\ceil{\frac{(n-1)}{2}}$ viele Knoten heraus. \\
			Da $n - (1+{\ceil{\frac{(n-1)}{2}})} \leq n/2$ wird die Anzahl der Knoten in jedem Schritt mindestens halbiert.
		Die maximale Anzahl an notwendigen Halbierungsschritten x für einen Turniergraphen der Größe n ergibt sich durch umformen der Gleichung $n \cdot \frac{1}{2}^x = 1$, da der Algorithmus spätestens bei einem Graphen der Größe 1 terminiert. Da $x=log(n)$ gilt impliziert dies, dass das für jeden Turniergraphen ein Dominating Set der Mächtigkeit log(n) existiert.
 
		\subsection*{c)}
			Da wir wissen , dass wir maximal log(n) Elemente brauchen , um ein Dominating Set für den Graphen zu erstellen ,sei M eine DTM die folgende Arbeitsweise hat:

			\begin{enumerate}
				\item M wählt in aufsteigender Reihenfolge maximal $log(n)$ viele Elemente aus n vielen Elementen
				\item Dann versucht sie nachzuweisen dass die gewählte Menge eine Dominating Set für den Grpahen darstellt. Wenn das ihr gelingt hält sie , wenn nicht geht sie weiter.
			\end{enumerate}

			Somit ist die Anzahl an Rechenschritten :

			\begin{align*}
				\sum_{i=1}^{\ceil{log(n)}} \binom{n}{i} &= \frac{n!}{(n-1)!} + \frac{n!}{(n-2)!2!}+\cdots+\frac{n!}{(n-\ceil{log (n)})!\cdot \ceil{log(n)}} \\
				&= n + \frac{n * (n-1)}{2} + \dots + \frac{n \cdot (n-1) \cdot \dots \cdot (n- \ceil{log n}+1)}{1 \cdot 2 \dots \ceil{log n}} \\
				&\leq n + n\cdot (n-1) + \dots + n \cdot (n-1) \dots (n- \ceil{log n} +1) = n^{O(log n)} 
			\end{align*}

			Die letzte Schranke setzt hingegen $n \geq 2$ voraus. Es ist aber sinnvoll das anzunehmen.
			Für $n=1$ sind wir sowieso in O(1) fertig. 
			

\end{document}
