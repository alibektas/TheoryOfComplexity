\documentclass{article}

\usepackage{amsmath}
\usepackage{enumitem}

\newlist{legal}{enumerate}{10}
\setlist[legal]{label*=\arabic*.}

\title{Überlegungen}
\author{Ali Bektasch}
\begin{document}
	\maketitle
	\section*{A)}
	\section*{B)}
		\begin{center}
		$\subset$
		\end{center}
		\begin{legal}
			\item Es seien p und q Polynome und  A $\in \oplus \oplus$ C ,  A' $\in \oplus$ C p-balancierte , \\
			 A'' $\in \oplus$ C q-balancierte Sprachen.
			\item C , $\oplus$ C , $\oplus \oplus$ C sind unter $\leq^{log}_m$ abgeschlossen. (siehe : Satz 87(i)).
			
			\begin{legal}
				\item Nach Lemma 86 folgt : \\
				Falls die Sprache B $\in$ C eine p-balancierte Sprache sein soll , dann existieren für jede Funktion f $\in $ FL ein Polynom q und q-balancierte Sprachen B',B'' in derselben Sprachklasse C mit

				\[
					\#B'(x) = \#B(f(x))  \text{ und } \#B''(x)/2^{q(|x|)} = \#B''(x)/2^{p(|f(x)|)}
				\]

				\begin{legal}
					\item Das ist für B$ \in \oplus \oplus $ C noch nicht ersichtlich , da sie keine p-balancierte Sprache ist.(nicht unbedingt!)
				\end{legal}
			\end{legal}

			\item Wir wollen zeigen : A $\in \oplus$ C. Die 

			\begin{legal}

			\end{legal}

		\end{legal}
	
\end{document}