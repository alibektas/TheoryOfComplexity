\documentclass{article}

\usepackage{amsmath}
\usepackage{amssymb}
\usepackage{german}

\def\mathify#1{\ifmmode{#1}\else\mbox{$#1$}\fi} % guarantee math mode
\newcommand{\ceil}[1]{\mathify{\left\lceil {#1}\right\rceil}}


\title{Übungsblatt 7}
\author{Ali Bektas 588063 \and Julian Kremer 562717 \and Ruben Dorfner 550204}


\begin{document}

	\maketitle
	
	\section*{Aufgabe 37}
		\subsection*{a)}
			\subsubsection*{$\subset$}
				\par Diese Richtung ist klar denn der maximale Weg zwischen zwei unter n Knoten , ohne Zyklen zu erzeugen , ist n-1 lang.
			\subsubsection*{$\supset$}
				\par Diese Richtung ist auch leicht zu zeigen. Wir nehmen ein bel. Element $y$ aus $(E\cap Id)^{n-1}$.Für dieses Element $y$ gilt entweder $y=(x,x)$ oder es gibt einen Weg der Länge maximal n-1 mit anderen Worten : y war entweder in E enthalten oder y ist durch Konkatinierung von zwei Pfäden entstanden.
		\subsection*{b)}
				\par Um eine \textbf{REACH}-Frage zu beantworten braucht man zunächst $E^{*}$.

				\textbf{Idee} : Multipliziere die Adjazenzmatrix $log n$ Mal mit sich selbst , also
				\begin{align*}
				&A = (E \cup Id)\\
				&\text{for log n times} :\\ 
				&\hspace{20px}A = A * A
				\end{align*} 
				Dadurch würden wir $(E \cup Id)^{n}$ berechnen was $E^*$ ist.

				Wir betrachten nun die Matrixmultiplikation.

				\[
					(AB)_{ij} = \lor_{k \in \{ 1,\dots ,n \}} (A_{ik} \land B_{kj})
				\]

				Die Verundung hat die Tiefe 1 und die Veroderung von $n$ Werten hat die Tiefe $log n$ , denn ODER-Gatter hat zwei Fanins.

				Es ergibt sich also $O(log^2 n)$.
		\subsection*{c)}
				Idee : Reduziere die Frage , ob ein Wort aus einer Sprache $L$ mit \[L\in NSPACE(s(n))\] auf die Frage von \textbf{REACH}. Dies gelingt einem , wenn man einen Graphen erstellt , dessen Knoten alle möglichen Konfigurationen sind. Eine NTM die die Sprache L entscheidet, kann maximal $2^{O(s)}$ Konfigurationen besitzen. Jetzt heißt die Frage : Kann man ausgehend von der Startkonfiguration in eine Endkonfiguration mit akzeptierendem Zustand gelangen? Wir können insgesamt $2^{O(s(n))}$ viele solche Fragen stellen und um jede solche Frage zu beantworten bräuchten wir einen Schaltkreis der Tiefe \[O(log^2 2^{O(s(n))}) = O(s(n)^2)\]

\end{document}
