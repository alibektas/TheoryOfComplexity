\documentclass{article}

\usepackage{amsmath}
\usepackage{amssymb}
\usepackage{stmaryrd}
\usepackage{german}


\def\mathify#1{\ifmmode{#1}\else\mbox{$#1$}\fi} % guarantee math mode
\newcommand{\ceil}[1]{\mathify{\left\lceil {#1}\right\rceil}}


\title{Übungsblatt 11}
\author{Ali Bektas 588063 \and Julian Kremer 562717 \and Ruben Dorfner 550204}


\begin{document}

	\maketitle

	\section*{Aufgabe 53}

		\subsection*{a)}
			\subsubsection{$\rightarrow$}
				
			\subsubsection{$\leftarrow$}
				A habe eine $\land_2$ in FL.
				Seien X und Y Sprachen aus C. Diese sind also auf A reduzierbar . Die Reduktionen erfolgen wiederum in FL(bezeichne die entsprechenden Funktionen mit f und f').

 				Betrachte die Fkt $f''$
				\[
					f''(f(e)\#f'(e)) \in A \iff f(e) \in A \land f'(e) \in A \iff e \in X \cap Y
				\]

				D.h f'' ist eine FL Reduktion auf A.Somit gehört der Durchschnitt auch zu der Sprachklasse C.

		\subsection*{b)}
			\subsubsection*{NP}
				Sei X und Y NP-Sprachen. Dann gibt es $N_x$ und $N_y$ die diese entscheiden. Sei N eine NTM die folgendes tut. 

				N simuliert Reihe nach $N_x$ $N_y$. Wenn N das Wort akzeptieren sollte , wenn eine der zu simulierenden Mascinen das Wort akzeptiert , so erkennt sie die Sprache $X \cup Y$. Wenn sie nur dann akzeptiert wenn beiden Maschinen akzeptieren so erkennt sie $X \cap Y$.
			\subsubsection*{co-NP}
				Vollkommen analog.
			\subsubsection*{$NP \cap co-NP$}
				Hier ist das Vorgehen wie folgt: Für jede Sprache $X\in NP \cap co-NP$ gibt es jeweils eine NP und co-NPTM. Dann sollte eine NP Maschine für $X\cap Y$(o. $ X \cup Y$) deren NP Maschinen wie bei NP (oben) nacheinander simulieren , und für co-NP Maschinen deren co-NP Maschinen 
			\subsubsection*{PH}
				Dies ist auch klar , denn wenn wir von Maschinen wie die im Satz 56 ausgehen würden und Zertifikate als Rechnung benutzen würden , ist es leicht zu sehen dass die Vereinigung oder Schnitt von zweier Sprachen wiederum in derselben Stufe enthalten ist. 
		\subsection*{c)}

\end{document}
