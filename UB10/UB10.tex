\documentclass{article}

\usepackage{amsmath}
\usepackage{amssymb}
\usepackage{stmaryrd}
\usepackage{german}


\def\mathify#1{\ifmmode{#1}\else\mbox{$#1$}\fi} % guarantee math mode
\newcommand{\ceil}[1]{\mathify{\left\lceil {#1}\right\rceil}}


\title{Übungsblatt 10}
\author{Ali Bektas 588063 \and Julian Kremer 562717 \and Ruben Dorfner 550204}


\begin{document}

	\maketitle

	\section*{Aufgabe 45}

		\subsection*{a)}Idee: Ordne jedem Buchstaben aus dem Alphabet eine Primzahl und kodiere jedes Wort aus der Sprache durch eine Sequenz von Primzahlen. Jedes Wort wird folglich als eine Primfaktorzerlegung dargestellt. Diese Zerlegung hat die Eigenschaft dass die Ordnung nicht wichtig ist , weshalb alle Permutationen eines Wortes auf dasselbe Tallywort abbilden.





\end{document}