\documentclass{article}

\usepackage{amsmath}
\usepackage{amssymb}
\usepackage{german}

\def\mathify#1{\ifmmode{#1}\else\mbox{$#1$}\fi} % guarantee math mode
\newcommand{\ceil}[1]{\mathify{\left\lceil {#1}\right\rceil}}


\title{Übungsblatt 8}
\author{Ali Bektas 588063 \and Julian Kremer 562717 \and Ruben Dorfner 550204}


\begin{document}

	\maketitle
	
	\section*{Aufgabe 39}
		Idee : In jedem Schritt führt das Flippen eines Literals in einer \textbf{unerfüllten} Klausel dazu , dass wir uns entweder (mit einer größeren Wkt. 'mehr dazu unten') h annähern oder davon wegbewegen.

		\subsection*{i)}
			$t_0 = 0 $ lässt sich wie folgt interpretieren : Wenn sich unsere Antwort von einer erfüllenden Belegung durch keine Variablen unterscheidet , dann ist die erwartete Anzahl an Schleifendurchläufen Null , was ja der Fall ist , denn wir haben keine zu erfüllende Klauseln , d.h die gesamte Formel ist erfült , woraus folgt dass While-Loop nicht durchlaufen wird.

		\subsection*{ii)}
			$t_n \leq t_{n-1} + 1$ : Wenn alle Variablen falsch belegt sind , dann sind wir nach einem Schritt näher an die erfüllende Belegung , weil es einfach nicht möglich ist mehr als die gesamte Anzahl von Variablen ($n$) falsche Belegungen zu haben.

		\subsection*{iii)}
			Da wir jedesmal eine falsche Klausel wählen und da die falsche Klauseln niemals 2 richtig gewählte Literalen enthalten können , hat man eine Wkt. von mehr als $\frac{1}{2}$ dass man nach diesem Schritt näher an die Lösung kommt. Wir betrachten die Random-Walk mithilfe von folgender Formel:

				\[ t_i = p_{i,i-1}(1+t_{i-1}) + p_{i,i+1}(1+t_{i+1}) \]


			wobei $p_{i,i-1}$ die Wkt. ist dass wir ein falsch belegtes Literal wählen. Wir setzen $p_{i,i-1} = \frac{1}{2} + \epsilon $ , mit $ \frac{1}{2} \geq \epsilon > 0 $.

			\begin{flalign*}
				t_i &= (\frac{1}{2} + \epsilon )(t_{i-1} +1) + (\frac{1}{2} - \epsilon)(t_{i+1} + 1)\\
				&= 1 + \frac{1}{2}t_{i-1} + \frac{1}{2} t_{i+1} + \underbrace{\epsilon t_{i-1} - \epsilon t_{i+1}}_{<0}\\ 
				&\leq 1 + \frac{1}{2}t_{i-1} + \frac{1}{2} t_{i+1}
			\end{flalign*} 

		\subsection*{iv)}
			Wir machen die Ungleichungen aus i,ii,iii zu Gleichungen. Die daraus gebildete Funktion stellt das Maximum dar. Dann

			\begin{flalign*}
				2t(1) - 2 &= t(0) + t(2)\\
				2t(2) - 2 &= t(1) + t(3)\\
				\vdots
				&\rightarrow 2(t(1) + \dots + t(n-1) - (n-1)) = t(0) + t(1) + 2(t(2)+ t(3) + \dots)\\
				&\rightarrow t(1) - 2(n-1)  + t(n-1) = t(n) = t(n-1) \\
				&\rightarrow t(1) -2(n-1) = 1
			\end{flalign*}

			Dadurch dass wir oben $t(1) = 2n -1 $ einsetzen erhalten wir im Allgemeinen $t_i = i(2n - i)$ und dadurch haben wir das Gesuchte. 

		Damit eine Eingabe von PM akzeptiert wird muss die Akzeptanzwkt bei korrekter Eingabe über $\frac{1}{2}$ liegen. Um zu argumentieren dass die Laufzeit dann polynomiell wäre , verwenden wir die Markov-Ungleichung. Aus iv folgt dass die erwartete Anzahl von Schleifendurchläufen $\leq n^2 $ist. Dann folgt:

		\[
			P(X \geq \frac{1}{2}) \leq \frac{n^2}{\frac{1}{2}} = 2n^2
		\]
		
\end{document}
