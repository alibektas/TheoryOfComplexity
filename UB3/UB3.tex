\documentclass{article}

\usepackage{amsmath , amsthm}


\setlength{\parindent}{15pt}

\def\mathify#1{\ifmmode{#1}\else\mbox{$#1$}\fi} % guarantee math mode
\newcommand{\ceil}[1]{\mathify{\left\lceil {#1}\right\rceil}}


\title{Übungsblatt 3}
\author{Ali Bektas 588063 \and Julian Kremer 562717 \and Ruben Dorfner 550204}

\begin{document}
\maketitle

\section*{Aufgabe 16}
\vspace{10px}

Wir zeigen zunächst die Aussage $A \in NE \Leftrightarrow NP$.\\
$\implies :$
Sei A eine Sprache  $ \subset \{0,1\}^*$. Es gelte $A \in NE$.

\begin{proof}

\[ A \in NE  \Leftrightarrow \exists N \in TM_{NTM} : N(x) , max\{ i | K_{start} \rightarrow^i_N K_{end}\}  \leq 2^{O(|x|)} \\
\]

Wir betrachten , 
was es für ein Wort aus tally(A) bedeutet , in NP-Time entschieden zu werden. 
Sei $n$ die Länge eines beliebigen Wortes aus  A. Dann  , zum Teil wegen der vorangestellten Eins und zum Teil wegen des exponentiellen Wachstums der Länge bei der Umwandlung von Binär- zu Unärdarstellung , lässt sich die Länge des Wortes tally(x) bzgl. x wie folgt erklären:

\begin{align}
2^{|x|} \leq |tally(x)| \leq 2^{|x|+1}
\end{align}


Wir wollen zeigen $ tally(A) \in NP $.  Dann bedeutet das in Anbetracht des (1) , dass es eine NTM N' geben muss , so dass sie tally(A) innerhalb von $NTIME(poly(n))$ entscheidet. Wir schreiben diese Aussage bzgl (1) um zu: $NTIME(poly(2^{|x| +1})) = NTIME(2^{c\cdot|x|+c}) = NTIME(2^{O(|x|)})$ wobei $x \in A$.

$A \in NTIME(2^{O(|x|)})$ ist die linke Seite der Aussage. Eine Maschine die die Sprache entscheidet , steht also zur Verfügung. Wir müssen nur noch zeigen dass es eine Routine gibt , die eine Eingabe in Form von tally(A) zu Binär umwandelt und diese in  $NTIME(2^{O(|x|)})-time$ schafft : 


\subsubsection*{Unär $\rightarrow$ Binär}

Um die Zeitbeschränktheit von einem solchen zu beurteilen schauen wir uns die $2^0$-te Stelle der Zahl , weil ihre Überquerungsfolge am längsten sein wird. Beim Hochzählen wird die erste Stelle 

\[ 2 \sum_{i=1}^{\ceil{log |x|}} \sum_{j=1}^{i} j \] 

mal überquert.

\begin{align*}
&=1 + 1+ 2 + 1 + 2 + 3 + \dots + 1 + 2 + ... + \ceil{log |x|}\\
&=2( \ceil{log n} + (\ceil{log n} -1) \cdot 2 + ... )\\
&=O(log(n))\\
\end{align*}


Die andere Richtung ist vollkommen analog und die Aussage $A \in E \Leftrightarrow tally(A) \in P$ ist nur eine schärfere Form der bewiesenen Aussage. Wir zeigen nur noch dass die Maschine die ein Wort aus A zu tally(A) umwandeln soll ,das innerhalb Zeit $O(poly(n))$ Zeit schaffen  kann .


\subsubsection*{Binär $\rightarrow$ Unär}
Sei N eine 3-NTM. Die Arbeitsweise zur Umwandlung ist wie folgt:

Initialisierung:
Kopf des 1. Bandes bewegt sich an das Ende der Eingabe.
Kopf des 2. Bandes schreibt eine 1.

Wenn der Kopf am ersten Band eine 1 liest , dann schreibt er soviel Nullen an das 3. Band wie es am 2. Band gibt , wenn nicht dann schreibt er nichts.
Wenn der Kopf am Band 1 sich nach links bewegt , dupliziert sich die Anzahl der Nullen am 2 Band.

Hierbei ist die Duplizierung die zeitaufwändigste Operation und sie kostet $\sum_{i=1}^{n}  i^2 = O(n^3)$ Also die Umwandlung gelingt ihr in poly. Zeit.

\subsection*{Schluss}
Nun sei A so eine Sprache dass für sie gilt : $A \in NE  \land  A\notin  E$. Dann:

\[ ( (A \notin E \implies tally(A) \notin P) \land (A \in NE \implies tally(A) \in NP) ) \implies P \neq NP \] 


\end{proof}




\end{document}  