\documentclass{article}

\usepackage{amsmath , amsthm}


\setlength{\parindent}{15pt}

\def\mathify#1{\ifmmode{#1}\else\mbox{$#1$}\fi} % guarantee math mode
\newcommand{\ceil}[1]{\mathify{\left\lceil {#1}\right\rceil}}


\title{Übungsblatt 3}
\author{Ali Bektas 588063 \and Julian Kremer 562717 \and Ruben Dorfner 550204}

\begin{document}
\maketitle

\section*{Aufgabe 16}
\vspace{10px}

Wir zeigen zunächst die Aussage $A \in NE \Leftrightarrow NP$.\\
$\implies :$
Sei A eine Sprache  $ \subset \{0,1\}^*$. Es gelte $A \in NE$.

\begin{proof}

\[ A \in NE  \Leftrightarrow \exists N \in TM_{NTM} : N(x) , max\{ i | K_{start} \rightarrow^i_N K_{end}\}  \leq 2^{O(|x|)} \\
\]

Wir betrachten , 
was es für ein Wort aus tally(A) bedeutet , in NP-Time entschieden zu werden. 
Sei $n$ die Länge eines beliebigen Wortes aus  A. Dann zum Teil wegen der vorangestellten Eins und zum Teil wegen des exponentiellen Wachstums der Länge bei der Umwandlung von Binär- zu Unärdarstellung , lässt sich die Länge des Wortes tally(x) bzgl. x wie folgt erklären:

\begin{align}
2^{|x|} \leq |tally(x)| \leq 2^{|x|+1}
\end{align}

\vspace{30px}
\textit{ Warum habe ich sowas geschrieben? Stellt euch vor : Wir haben eine Binärzahl 1001 => 9. Wir stellen eine 1 voran. So wird die Zahl 11001 => $2^4 + Alte Zahl = 2^|x| + Alte Zahl $. Den Rest kriegt ihr hin.}
\vspace{30px}


Wir wollen zeigen $ tally(A) \in NP $.  Dann bedeutet das in Anbetracht des (1) , dass es eine NTM N' geben muss , so dass sie tally(A) innerhalb von $NTIME(poly(n))$ entscheidet. Wir schreiben diese Aussage bzgl (1) um zu: $NTIME(poly(2^{|x| +1})) = NTIME(2^{c\cdot|x|+c}) = NTIME(2^{O(|x|)})$ wobei $x \in A$.

Das zuletzt hergeleitete ist der linken Seite gleich , somit ist diese Richtung bewiesen.

Die andere Richtung ist vollkommen analog und die Aussage$A \in E \Leftrightarrow P$ ist nur eine schärfere Form der bewiesenen Aussage.

Nun sei A so eine Sprache dass für sie gilt : $A \in NE  \land  A\notin  E$. Dann:

\[ ( (A \notin E \implies tally(A) \notin P) \land (A \in NE \implies tally(A) \in NP) ) \implies P \neq NP \] 

\end{proof}

\#TODO
Man muss noch erwähnen warum die Umwandlungsmechanismen von Binär zu Unär und umgekehrt keine Rolle für die Laufzeit spielt.(Weil sie einfach sind). Dann ist diese Aufgabe auch fertig.




\end{document}  